\section{The website}
\label{sec:website}

This section describes the main technical characteristics of the website, and the technologies it is built with.
It starts with the notion of Single-page application, that is explained in {\sc subsection}~\ref{ssec:spa}.

\subsection{Single-page application}
\label{ssec:spa}

\url{www.konnektid.com} is a special kind of web application called Single-page application (\guillemotleft{} SPA \guillemotright{}).
This means that it loads only one web page, and the totality of its necessary code, right from the start.
After that initial load, all of the presentation logic is handled by the client side, through Javascript.
When the user interacts with the page, only affected parts of it are dynamically updated, without any server intervention.

This introduces the notion of \guillemotleft{} Views \guillemotright{}: they are like different versions of the same page, that differentiate by one or several HTML fragment(s).
They appear as soon as the user interacts with the page, and sometimes several are loaded at the same time.
They usually offer a set of functionalities, for instance a form to fill or a menu.

Views have a direct impact on the DOM~\footnote{Document Object Model}, which provides a structured representation of the web page's content (elements, styles\ldots).
In SPAs, the DOM is usually write-only, meaning that it cannot store any data or provide any information.
All it does is updating itself as soon as a new view appears, in order to display the correct elements.

The main advantage of SPAs is to create a more fluid and responsive user experience, by avoiding page reloads.

There are various Javascript techniques and frameworks available for implementing SPAs.
Among them, Konnektid uses several that are described in {\sc subsection}~\ref{ssec:frameworks}.

\subsection{Frameworks and libraries}
\label{ssec:frameworks}

Languages: Javascript, HTML, SCSS\ldots

\begin{itemize}
    \item ReactJS
    \item NodeJS
    \item Redux
    \item GraphQL
\end{itemize}
