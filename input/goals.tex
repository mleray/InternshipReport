\section{Goals of the internship}
\label{sec:goals}

The main goal of the internship was the implementation of new pages and features for the platform, needed to prove Konnektid's business model.
For this same purpose, some of the original elements required improvements or refactoring. And as in any website, there were also a few bugs to fix.
From a personal point of view, these various tasks were meant to discover web development and the related technologies: this is elaborated in {\sc subsection}~\ref{ssec:web}.

\subsection{Web development}
\label{ssec:web}

Most of the achievements of the traineeship were directly linked to the website, so it was a great way to discover and learn web development.
\url{www.konnektid.com} is mostly built in JavaScript, HTML and SASS~\footnote{An extension of CSS, more powerful than the basic language.}
so those programming languages, almost unknown at first, needed to be mastered.
Although a major part of the assigned tasks were focused on front-end development,
it was expected to work on back-end duties from time to time, so in general to be flexible and eager to learn.

Konnektid uses some of the most recent web development and JavaScript tools available, such as ReactJS, NodeJS, Redux, and more.
They are presented in details in {\sc section}~\ref{sec:website}.

There was another intention to this internship, which is described in {\sc subsection}~\ref{ssec:companyBuilding}:
understanding the whole company building process.

\subsection{Company building}
\label{ssec:companyBuilding}

Being part of a startup team is a rich experience for many reasons, and one of them is learning the challenges of building a company.
Even as an intern, it was expected to be part of the process, and not in a passive way:
indeed, it is highly appreciated to be able to give ideas and opinions.
This could happen at several occasions, including brainstorms, meetings and demonstrations of new features.
The project management and team organization are explained in {\sc section}~\ref{sec:management}.

Moreover, as resources are limited and the number of employees is quite small, it was supposedly fast to be in the position of decision making.
This teaches independence and autonomy, but also implies to take responsibility for choices and actions that were made.
