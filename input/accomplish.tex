\section{Internship accomplishments}
\label{sec:accomplish}

This section describes the main tasks achieved during the traineeship, along with the reason why they were needed, the way they have been implemented and the knowledge that they brought.

\subsection{Building of UI components}
\label{ssec:ui_components}

As mentioned in {\sc subsection}~\ref{ssec:frameworks}, ReactJS is based on encapsulated components, most of them appearing several times on the website. They can be very basic (e.g. buttons) or more elaborate (e.g. modals). For this reason, it is a good idea to create a library of generic components, that can be reused as many times as required. This saves a lot of development time, and also assures a certain homogeneity between the different pages of the application.

Several developers, including myself, built this library in a folder named \guillemotleft{} ui \guillemotright{}. One of the components that I have created is the \guillemotleft{} DatePicker \guillemotright{}. Since this is a rather complex component, a good reflex is to first take a look at existing ReactJS plugins. They are usually hosted on Github, with a demonstration link of how they behave and a quick download option via npm.

I found a few possible date pickers, so a selection had to be made, based on several criteria. One of them is the latest update of the repository: if it is quite recent, it means that the owner can quickly help in case there is any problem with the plugin. Also, if the repository has a lot of stars, it means that lots of people using it are satisfied. Another good sign is a low number of open issues in the repository. In the end, the choice was a plugin called \guillemotleft{} react-datepicker \guillemotright{}~\footnote{https://github.com/Hacker0x01/react-datepicker}.

Then, to start the implementation, a \guillemotleft{} DatePicker \guillemotright{} folder was added to the library of components. In this new folder, I created a file named \textit{DatePicker.js} which contains the code of the component. At the top of this file are a few \guillemotleft{} import \guillemotright{} statements, as seen on {\sc figure}~\ref{fig:imports}: ReactJS, the previously downloaded plugin, and also \textit{DatePicker.scss}.

\begin{figure}[H]
    \centering
    \includegraphics[scale=0.9]{figure/imports.png}
    \caption{The needed importations for the DatePicker UI component.}
    \label{fig:imports}
\end{figure}

This last imported file contains the local CSS styles for the date picker, that will not be used anywhere else in the application. This system makes sure that if changes are made in the CSS for one component, they will not affect other UI elements. However, Konnektid uses some global CSS values (fonts, colors, etc) that are defined in SCSS files at the root of the \guillemotleft{} ui \guillemotright{} folder. Those are meant to be reused in every component, to make the website more harmonious. So I used them to customize react-datepicker, and the final result is presented on {\sc figure}~\ref{fig:datePicker}.

\begin{figure}[H]
    \centering
    \includegraphics[scale=0.6]{figure/datePicker.png}
    \caption{The final DatePicker presentational component.}
    \label{fig:datePicker}
\end{figure}

This assignment was an efficient way to learn JavaScript and some good ReactJS practices. Moreover, it was a great introduction to the ReactJS community and all the support that it brings, especially with the numerous plugins that are shared.

\subsection{Implementation of new pages}
\label{ssec:new_pages}

It has been explained in {\sc subsection}~\ref{ssec:concept} that Konnektid's main goal for the year is to test and validate their business model, which relies on
professional teachers. This requires the implementation of several new pages and functionalities, and some have been built during the internship. Among them, only
two will be described here: first the course page, then the teachers landing page.

\subsubsection{The course page}
\label{sssec:coursePage}

The course page refers to the page used by teachers to create, edit and publish a course. It had to be built from scratch, for both desktop and mobile, based on designs
made by Konnektid's former designer. It is now released and used by professional teachers, and a desktop example is visible in {\sc attachment}~\ref{ssec:courseDesktop}.

The page features two main elements:

\textbf{The top section} which contains all the main information about the course (picture, title, price\ldots) and the teacher (avatar, name, short introduction).
The teacher name is a direct link to his or her profile. On the right, this section also provides a button \guillemotleft{} Enroll now \guillemotright{} for the students
to book the course.

\textbf{The bottom section} which is divided in two parts. The right part takes most space on the page and gives an in-depth description of the course
(methodology, requirements\ldots). The left part, which is thinner, shows practical information about the course (format, length\ldots) and another
\guillemotleft{} Enroll now \guillemotright{} button with a reminder of the price. Below this are the sharing functionalities: Facebook, Twitter,
email, and WhatsApp on mobile.

The first step in the implementation was to create the corresponding route, to indicate that \url{www.konnektid.com/course/:id} (where \textit{id} is the course identifier) leads to the \textit{Course} route handler. It is a JavaScript file rendering the \textit{Course} component inside of the website layout.

Then, to build the \textit{Course} component, the challenge was to decide how to divide
it in sub-components, each of them implemented in their own folder with local CSS styles. For instance, the biggest bottom part is a component called
\guillemotleft{} CourseDescriptionCard \guillemotright{}. It contains several sections that are similar in presentation and only differ by their title and content, so
it made sense to create a reusable \guillemotleft{} CourseDescriptionItem \guillemotright{} to render each of them with title and content passed down as props. It respects the single responsibility principle previously mentioned in {\sc subsection}~\ref{ssec:frameworks} and avoids duplicated code.
These are two of the many good practices for writing maintainable code~\cite{maintainable}, i.e. code that is easy to read, to modify and to extend.

After the static course page had been built and reviewed, my tutor asked me to add inline editing features to it.
This means that elements can be edited in-place, in the context where they will be published, which facilitates picturing the final result.
In order to implement this, I used an editor framework for ReactJS called \textit{DraftJS}, which enables the creation of rich content such as bold/italic text or lists of items~\cite{draftJS}. The result for the top section of the page is displayed on {\sc figure}~\ref{fig:courseEditIntro}. When hovering on an editable part, such as the title, a gray frame appears around it to draw attention, and the teacher can just click on it to start writing.

\begin{figure}[H]
    \centering
    \includegraphics[scale=0.2]{figure/courseEditIntro.png}
    \caption{The top section of the course page in edition mode.}
    \label{fig:courseEditIntro}
\end{figure}

DraftJS is based on an \guillemotleft{} Editor \guillemotright{} component, whose state is stored in an \guillemotleft{} EditorState \guillemotright{} object holding all the information about the content: the text and its decoration (bold, italic\ldots), the selection state\ldots A method \textit{onChange}, passed down as prop, is called to update the state every time the user changes the content of the editor. Moreover, it is possible to define customized key bindings for the editor, by using the \textit{keyBindingFn} and \textit{handleKeyCommand} props. The first function returns a command string depending on the pressed key, then the second one takes that string as input and outputs the corresponding changes on the editor state.

This page was the first static page I fully created, and it made me realize how useful the components library mentioned in {\sc subsection}~\ref{ssec:ui_components} is. It was also a great occasion to improve my skills in JavaScript and SCSS, and to get even more familiar with ReactJS and its derivatives.

\subsubsection{The teachers landing page}
\label{sssec:teachersPage}

This is the page describing Konnektid's offer for professional teachers, in order to convince people to become one. It was already existing, but it was decided to improve it because the conversion rate (i.e. the number of users signing up as teachers compared to the number of visitors of the page) was too low. 

To get started, we had a brainstorm session to determine what contents were needed, and how they should be structured. We came up with a wireframe\footnote{Basic skeleton of the page, representing the main UI elements and how they work together.}, and since there was no designer back then, this was the only support I had for development. So I quickly built and presented a first draft, to make sure I was heading the right way and to get a first round of feedback.

Based on this, I could iterate the process and gradually improve the interface until it was validated, for both desktop and phone. Then, to verify that we were sending the right message, the copywriting was reviewed together with the community manager. She also helped me out by collecting quotes from current professional teachers, who were then integrated in the page as social proof of the concept. This first version was released in \textit{alpha}. A while later, a designer joined the team and suggested a few modifications concerning the User Experience (UX) and the UI of the page, so we worked together on improving it. 

In the meantime, the registration process for professional teachers was also being refactored, but it was not yet ready. So I had to implement a temporary sign-flow, based on the former one visible on {\sc figure}~\ref{fig:signUpFlow}.

 \begin{figure}[H]
    \centering
    \includegraphics[scale=0.6]{figure/signUpFlow.png}
    \caption{The registration form for professional teachers.}
    \label{fig:signUpFlow}
\end{figure}

It is a modal with four fields to fill in: the name and email of the applicant, the skill(s) to be taught, and the city or cities where he/she could teach. When clicking on \guillemotleft{} Send \guillemotright{}, an email is sent to the Konnektid team with all the provided information. If everything goes well, a new screen appears to indicate the success, and the request can be treated by Konnektid (i.e. get in touch with the person, verify the certifications\ldots). If there is a problem (for instance while sending the email), an error screen appears to inform the applicant that the request could not be sent.

I had to re-create this modal with the UI components of the new library, as a component called \guillemotleft{} BecomeTeacherModal \guillemotright{}. It has a corresponding container to manage the state of the modal (i.e. when to show it or not) and of the form (i.e. which screen to display) thanks to a Redux reducer. It also contains a form, the \guillemotleft{} BecomeTeacherForm \guillemotright{} component, receiving the screen as prop, which can have three values: \guillemotleft{} form \guillemotright{}, \guillemotleft{} success \guillemotright{} or \guillemotleft{} error \guillemotright{} depending on the state of the request. This form was built with \textit{redux-form}, a plugin specially designed for managing form state in React using Redux.

I used \textit{redux-form} in a container on top of the BecomeTeacherForm component, to perform all the necessary verifications needed to validate the form. For example, the email address is mandatory, and also needs to respect a certain format (e.g. test@test.com is good, but test.test.com is not because it does not contain @). The code of the BecomeTeacherForm container, testing the email value, can be seen on {\sc figure}~\ref{fig:reduxForm}. 

The final desktop version of the new teachers landing page is available in {\sc attachment}~\ref{ssec:teachersPage}. It was great to follow the creation process from A to Z, and to be fully responsible for it. Another highlight of this project was dealing with UX/UI design issues, because no pre-designed mockups were available but also thanks to the close collaboration with the designer at the end.

\subsection{Creation of flows}
\label{ssec:flows}

Matchmaking flow

\subsection{New navigation}
\label{ssec:new_nav}

\begin{itemize}
    \item Responsible of the project
    \item Added feature flag, new navigation in old pages for both mobile and desktop, only in desktop for new pages
\end{itemize}

\subsection{Analytics}
\label{ssec:analytics}

\begin{itemize}
    \item Updated analytics in old pages (for instance for inviting friends)
    \item Added analytics to new pages (explain necessary refactoring for enroll modal?)
\end{itemize}

In addition to the technical learnings, this internship has been a great occasion to dive into startup life and to discover the process of company building. This requires a solid project management system, and the one used at Konnektid is explained in {\sc section}~\ref{sec:management}.