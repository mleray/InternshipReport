\section{Company presentation}
\label{sec:company}

Konnektid is a startup, founded in January 2013 by Michel {\sc Visser}, the current CEO~\footnote{CEO: Chief Executive Officer},
with the goal of transforming neighborhoods into universities. A few months later, Simone {\sc Potenza} joined the project
as co-founder and CTO~\footnote{CEO: Chief Technology Officer}. The company's website, www.konnektid.com, was live in April 2014. It features a slogan, which is \guillemotleft{} Everyone has a skill worth sharing \guillemotright{},
and a logo visible on {\sc figure}~\ref{fig:logo}.
\vspace{1cm}

\begin{figure}[h]
    \centering
    \includegraphics[scale=0.6]{figure/logo_konnektid.png}
    \caption{Konnektid logo, representing conversation bubbles.}
    \label{fig:logo}
\end{figure}

\subsection{Current situation}
\label{ssec:situation}

\begin{itemize}
    \item 15,000+ members
    \item Mostly in the Netherlands
    \item Office at Rockstart (small intro)
    \item Team (small intro)
\end{itemize}

\subsection{Concept}
\label{ssec:concept}

Konnektid is a skill-sharing platform, empowering life-long learning and community building. It connects people who want to learn with people who can teach, and who live nearby.
This last condition is what makes Konnektid special: people find each other online, but then meet face-to-face to exchange their knowledge.
Behind this idea is a strong belief that skill-sharing is more efficient this way, with the possibilities of asking questions for the student and reformulating for the teacher.
On the website, these offline meetings are called \guillemotleft{} konnektions \guillemotright{}.

Two different kinds of members can be found in the community:

\begin{itemize}
    \item Professional teachers, who pay a monthly amount for using the platform;
    \item Regular members, who use the website for free.
\end{itemize}

Professional teachers pay a monthly fee to use the platform, and get an public profile featuring their name and location but also their experience and certifications.
They can create courses or workshops, which they charge for and are able to share on main social media. So Konnektid is providing them a marketing tool,
with access to a community of potential students, and also tips and help from the community manager.

On the other hand, regular members use the service free of charge. They own a private profile containing their name, their location and eventually some skills that they can teach.
They can send general requests, notifying their neighbours that they want to learn something specific.
If they see that somebody offers that skill, they can also send a personal request to that person.
And if they are willing to pay, they can book courses or workshops offered by professional teachers.

Goal for the year: validate business model and become sustainable (transition with next section: goals of the internship)
