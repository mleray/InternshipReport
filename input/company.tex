\section{Company presentation}
\label{sec:company}

Konnektid is a startup, founded in January 2013 by Michel {\sc Visser}, the current Chief Executive Officer (CEO),
with the goal of transforming neighborhoods into universities. A few months later, Simone {\sc Potenza} joined the project
as co-founder and Chief Technology Officer (CTO). The company's website, \url{https://www.konnektid.com/}, was live in April 2014.

\subsection{Concept}
\label{ssec:concept}

Konnektid is a skill-sharing platform, empowering life-long learning and community building. It connects people who want to learn with people who can teach, and who live nearby.
This last condition is what makes Konnektid special: people find each other online, but then meet face-to-face to exchange knowledge.
These offline meetings are called \guillemotleft{} konnektions \guillemotright{}.

The website features a slogan, which is \guillemotleft{} Everyone has a skill worth sharing \guillemotright{},
and a logo visible on {\sc figure}~\ref{fig:logoKonnektid}.
\vspace{1cm}

\begin{figure}[h]
    \centering
    \includegraphics[scale=0.6]{figure/logo_konnektid.png}
    \caption{Konnektid logo, representing conversation bubbles.}
    \label{fig:logoKonnektid}
\end{figure}

Two different kinds of members can be found in the community:

\begin{itemize}[noitemsep]
    \item Professional teachers, who pay a monthly amount for using the platform;
    \item Regular members, who use the website for free.
\end{itemize}

Professional teachers pay a monthly fee to use the platform, and get an public profile featuring their name and location but also their experience and certifications.
They can create courses or workshops, which they charge for and are able to share on main social media.
Plus, they have access to the community in order to find potential students.

On the other hand, regular members use the service free of charge. They own a private profile containing their name, their location and eventually some skills that they can teach.
They can send general requests, notifying their neighbours that they want to learn something specific.
If they see that somebody offers that skill, they can also send a personal request to that person.
And if they are willing to pay, they can book courses or workshops offered by professional teachers.

\subsection{Current situation}
\label{ssec:situation}

The Konnektid team has expanded, and it has changed a lot over the years with occasional freelancers and trainees.
At the moment it consists of a dozen of people, from very different origins and backgrounds, some of them working part-time or remote.

The website has also grown, and it now counts more than 15.000 members.
These users mostly live in the Netherlands, in the big cities such as Amsterdam, Rotterdam or Utrecht.

Konnektid's current office is located in the center of Amsterdam, in Rockstart's building.
Rockstart is a startup accelerator, meaning that they help startups kickoff, develop and grow through funding and mentorship.
It results in a great community of people with the same interests in technology and entrepreneurship, who are willing to help each other.

The notion of professional teachers described in {\sc subsection}~\ref{ssec:concept} is quite recent on the website: it is part of Konnektid's business model.
The company's major goal for this year is to test it, and to validate it in order to become sustainable.
This requires the building of many new functionalities, which was one of the main goals of the internship detailed in {\sc section}~\ref{sec:goals}.
