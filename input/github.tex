\section{Github}
\label{sec:github}

To manage and host the code of the web application, Konnektid's developers use Github. It is an application built on top of Git, a distributed version control system popular for its branching and merging functionalities~\cite{git}. As Git is originally based on command line, Github provides an interface to visualize all Git events such as commits, merges, and more.

\subsection{Environments}
\label{ssec:env}

Konnektid's website is deployed in three distinct environments, defined by the code itself but also by their unique configuration for the application. They are described in the following paragraphs.

\textbf{development} is a working directory for current tasks, each of them in a different local branch created from \textit{alpha}. Once the code is ready, a pull request is opened in order to let the other developers know that the feature is completed. Then, the branch is first merged in \textit{alpha} before being released in \textit{production}. 

\textbf{alpha} is a pre-release of new features, useful to provide further tests, fix remaining bugs, and make sure that the everything is compatible with \textit{production}. The rest of the Konnektid team also has access to it, so that they can see and try out the latest functionalities.

\textbf{production} contains the version published on \url{www.konnektid.com}.

When implementing an important modification in the code (for instance, change the general navigation), a \guillemotleft{} feature flag \guillemotright{} is used. It is a boolean, defined in the configuration files, that decides whether or not the new feature should be displayed in the branch. This trick assures that the critical changes will not affect the other branches.

A piece of code does not automatically move from \textit{development} to \textit{production}. First, it has to pass a certain number of tests before being committed and pushed. Then, the resulting pull request is manually reviewed: this is clarified in {\sc subsection}~\ref{ssec:reviewing}.

\subsection{Reviewing process}
\label{ssec:reviewing}

When a piece of code is ready to be merged, first in \textit{alpha}, the developer who implemented it adds a \guillemotleft{} state/need review \guillemotright{} tag to the corresponding pull request. Then, another member of the development team will take a look at the changes. This verification has three main goals:

\begin{itemize}[noitemsep]
	\item Verify that the changes fulfill their goal;
 	\item Check the quality of the code (for instance the indentation);
	\item Make sure that the merge will not cause any conflicts.
\end{itemize}

If something is wrong, the reviewer replaces the previous tag by a new one, \guillemotleft{} state/need improvement \guillemotright{} with a comment in the pull request to describe what needs to be enhanced. This allows the reviewee to efficiently iterate on it. Otherwise, if everything is fine, the reviewer adds a \guillemotleft{} state/approved \guillemotright{} tag and merges the branch in \textit{alpha}.

This process is rather efficient but very time-consuming, and not completely reliable as humans can make mistakes. To reduce this risk, Konnektid is introducing continuous integration (\guillemotleft{} CI \guillemotright{}) with Jenkins, an open source automation server~\cite{jenkins}. 

CI is a practice based on frequently integrating local code into the shared repository, where it is verified by an automatic build. It is meant to reduce integration issues, and if there are any it helps detecting them earlier. At the moment, CI in Konnektid is very basic and only performs a few tests on \textit{alpha}, but the goal is to improve it step by step in order to achieve continuous integration, continuous delivery (\guillemotleft{} CD \guillemotright{}) and fully automated testing. CD aims at safely releasing changes in \textit{production} as soon as a pull request is merged, so that it is faster to get feedback from the users and therefore easier to update.

Finally, Konnektid tracks many analytics from their website. Their implementation, purposes and associated tools are detailed in  {\sc section}~\ref{sec:analytics}.